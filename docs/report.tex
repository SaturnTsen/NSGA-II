\documentclass[english]{article}

% Font settings (for XeLaTeX or LuaLaTeX compilers)
\usepackage{fontspec} % Font settings, suitable for XeLaTeX / LuaLaTeX
\usepackage{amsmath, amssymb, mathtools, amsthm} % Mathematical symbols and environments
\usepackage{graphicx} % Graphics handling
\usepackage{tikz} % Drawing
\usepackage{xcolor} % Color management
\usepackage{hyperref} % Hyperlinks
\hypersetup{hidelinks} % Hide hyperlink borders
\usepackage{caption} % Captions for figures and tables
\usepackage{subcaption} % Subfigures
\usepackage{float} % Force figure/table placement
\usepackage{booktabs} % Table beautification
\usepackage{longtable} % Support for long tables
\usepackage{enumitem} % Customizable lists
\usepackage{fancyhdr} % Custom headers and footers
\usepackage{geometry} % Page layout
\geometry{left=0.9in, right=0.9in, top=1in, bottom=1in} % Page margins
\usepackage{setspace} % Line spacing
\setstretch{1.2} % Set line spacing

% Support for Chinese characters
\usepackage{xeCJK} % Support for Chinese, suitable for XeLaTeX compiler

% French support
\usepackage[english,french]{babel} % Support for English and French

% Custom titles
\usepackage{titling} % For customizing titles
\usepackage{tcolorbox} % Colored boxes
\usepackage{algorithm2e} % Algorithm framework
\usepackage{algorithmicx} % Algorithm environment

% Custom colors
\definecolor{urlcolor}{rgb}{0,.145,.698}
\definecolor{linkcolor}{rgb}{.71,0.21,0.01}
\definecolor{citecolor}{rgb}{.12,.54,.11}

% Set page headers and footers
\pagestyle{fancy}
\fancyhf{} % Clear headers and footers
\fancyhead[L]{Left Header} % Left header
\fancyhead[C]{Center Header} % Center header
\fancyhead[R]{Right Header} % Right header
\fancyfoot[C]{\thepage} % Centered page number in footer

% Other common settings
\newtheorem{theorem}{Theorem}[section] % Theorem environment
\newtheorem{lemma}[theorem]{Lemma} % Lemma environment
\newtheorem{corollary}[theorem]{Corollary} % Corollary environment

% Set colors for cross-references
\hypersetup{
    linkcolor=linkcolor,
    citecolor=citecolor,
    urlcolor=urlcolor
}

% Chinese font settings (can be changed as needed)
\setCJKmainfont{SimSun} % Set Chinese font (here using SimSun)
\setCJKsansfont{SimHei} % Set sans-serif font
\setCJKmonofont{FangSong} % Set monospaced font

% Citation handling
\usepackage{cite} % For handling multiple citations

\begin{document}

% TODO Write the report content here

\title{Evolutionary Multi-Objective Optimization}
\author{Yiming Chen, Linh Vu Tu}
\date{\today}

\maketitle

\begin{abstract}
    This is the abstract of the report.
\end{abstract}

\section{Introduction}
This is the introduction section. This is where we cite a paper \cite{example2025}.

\section{Methodology}
This is the methodology section. Another reference \cite{samplebook2024} is cited here.

Lorem ipsum


\subsection{Task 2}

The Pareto set of $LOTZ$ is defined by all of the individuals of the form $1^k 0^{n-k}$ where $k \in [0..n]$. Indeed, given $\alpha \in \{0, 1\}^*$ and $p, q \in \mathbb{N}$, $1^p 0 \alpha 0^q \prec 1^{p+1} \alpha 0^q$ and $1^p \alpha 1 0^q \prec 1^p \alpha 0^{q+1}$ with respect to $LOTZ$. Thus, the Pareto front of $LOTZ$ is $\{ (k, n - k) | k in [0..n] \}$. Both of these sets have cardinality $n + 1$.

These results easily generalize to the $mLOTZ$ objective function. Indeed, $mLOTZ$ simply computes $LOTZ$ on \textbf{disjoint} chunks of an individual. In other words, Pareto-optimal values are also optimal on the $m$ individual chunks of size $n'$. It follows that the Pareto set of $mLOTZ$ is simply the set of $m$-concatenations of individuals of size $n'$ in the Pareto set of $LOTZ$. Formally, we can write the Pareto set as $1^{k_1} 0^{n' - k_1} 1^{k_2} 0^{n' - k_2} \cdots 1^{k_m} 0^{n' - k_m}$ with $k_1, \cdots, k_m \in [0..n']$. The Pareto front is written as
$$\{ (k_1, n' - k_1, k_2, n' - k_2, \cdots, k_m, n' - k_m) | (k_1, \cdots, k_m) in [0..n']^m \}$$
Both of these sets have cardinality $(n' + 1)^{m / 2}$.

\subsection{Task 3}

Our algorithm for non-dominated sorting builds the directed acyclic graph (DAG) corresponding to the partial order defined by the multi-objective function. Since we use adjacency lists, the time complexity is $\mathcal{O}(N^2)$.

Iterating on the fronts (\code{Graph::pop_and_get_fronts}) is a flood-fill algorithm. Since we use a queue, this behaves like breadth-first search as every vertex and every edge is visited only once. The complexity of BFS is $\mathcal{O}(N + M) = \mathcal{O}(N^2)$.

Finally, the algorithm terminates and runs in $\mathcal{O}(N^2)$ time.

Proving the correctness of our algorithm requires one extra step. When iterating on a front $i$, we decrement the in-degree of all of the neighbors for each outgoing edge from $i$. We can prove by induction over the fronts that the vertices ending up with in-degree zero are exactly those in front $i+1$, which concludes the correctness of the algorithm.

\subsection{Task 4}

The crowding distance of an individual is the sum of its crowding distance relative to each objective. Since there are $m$ objectives, the time complexity of the function has a $\mathcal{O}(m)$ multiplicative factor due to the outermost loop.

Sorting the individuals in a population is performed in $\mathcal{O}(N log N)$ time, and computing the crowding distance of a single individual has $\mathcal{O}(1)$ runtime since evaluating $f$ is done in constant time.

Finally, the time complexity of computing the crowding distance of a population is $\mathcal{O}(m N log N)$.

\subsection{Task 6}

In our batched experiments, we find out that the Pareto front is systematically covered after a few epochs when the population is small ($N \le 1024$). We choose $m$ and $m$ so that $n > m$. Since we choose $N = 4 M = 4 \left(\frac{2 n}{m} + 1 \right)^{m / 2}$, $N$ grows exponentially with $m$, we had to constrain $m$ and $n$ to relatively small values.

\section{Results}
This is the results section.

\section{Conclusion}
This is the conclusion section.

% 引用文献
\bibliographystyle{plain} % 指定BibTeX引用样式(例如plain)
\bibliography{references} % 加载参考文献文件 references.bib

\end{document}
